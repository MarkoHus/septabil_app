\chapter{Zaključak i budući rad}
		
		
		
		 
		
		
		 
		 Zadatak naše grupe bio je izrada mobilne aplikacije za Android operacijski sustav koja bi ubrzala digitalizaciju računovodstvenim tvrtkama. Izrada aplikacije završena je u zadanom roku, a postupak izrade aplikacije se odvijao u 2 ciklusa.\\
		 \indent U prvom ciklusu nam je cilj bio napraviti jednostavnu verziju aplikacije kako bismo se upoznali s novim tehnologijama i alatima. Nakon okupljanja projektnog tima te nakon upoznavanja sa zadanim zadatkom, krenuli smo prikupljati literaturu te potrebne materijale koji su nam pomogli da se upoznamo s korištenim alatima i tehnologijama. Na prvom sastanku smo se podijelili u manje grupe te je svaka grupa dobila zadatak izrade određenog dijela aplikacije, što nam je uvelike olakšalo izradu same aplikacije. Kroz izradu alfa verzije aplikacije smo se upoznali sa novim alatima i problemima koje smo trebali rješiti, te je alfa verzija aplikacije i dokumentacija bila završena na kraju prvog ciklusa.\\
		 \indent U drugom ciklusu smo implementirali ostale funkcionalnosti aplikacije te ju uredili, reorganizirali bazu podataka i dodali nove lambda funkcije na back-endu. Nakon toga smo napisali i dodali svu potrebnu dokumentaciju.
		  \indent
		 Kroz proces izrade projekta bili smo suočeni s mnogim novim tehnologijama i problemima koji nam prije nisu bili poznati te smo ih sve uspješno rješili i sad smo bogatiji tim znanjem i kompetentniji.
		 Prvo pitanje koje smo rješili je bilo koju čemo infrastrukturu koristiti za izradu projekta, za što je bilo potrebno puno istraživanja raspoloživih opcija te bi rješenje puno brže našli da nam se ponovno postavi isto pitanje, jer smo upoznati sa terenom i opcijama. Isto tako je bilo i sa odabirom alatima koje smo koristili za izradu pojedinih dijelova projekta, kao što su Android studio, AWS-ove lambda funkcije i API gateway, no nakon što smo se upoznali samo je bilo potrebno proniknuti dublje u dostupna znanja alata/ tehnologije kojom se služimo i to sve uspješno povezati u jednu cjelinu sa komunikacijom među svojim dijelovima. Područje znanja koje bi nam nakon ovog projekta najviše pomoglo za kvalitetniji i efikasnije rješen slijedeći projekt je poznavanje Android studia, izrada moblinog dijela aplikacije i front-enda.
		 Nakon ovog projekta smo stekli vrijedno iskustvo izrade potpune aplikacije od početka do kraja sa server-sideom, te bi nam sada izrada nove mobilne aplikacije bila puno jednostavnija i lakša, za što bi se i odlučili da nam se ukaže prilika ili da pronađemo ideju vrijednu realizacije.
		  \indent
		 Funkcionalnosti koje bi još ostvarili da usavršavamo aplikaciju bi bilo dodavanje pop-up obavijesti naše aplikacije.
		 
		
		\eject 